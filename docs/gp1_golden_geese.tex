\documentclass[12pt]{article}
\usepackage{fancyhdr}
\pagestyle{fancy}
\rhead{User Guide}
\lhead{\today}
\cfoot{}
\renewcommand{\headrulewidth}{0.4pt}
\renewcommand{\footrulewidth}{0.4pt}
\title{Programming Fundamentals I\\ {\bf Group Project: The Goose Game, reimagined}}
\author{Paulo Aurecchia, Samuele Decarli, Irene Jacob, Julia Traiba}
\date{}

\begin{document}

\maketitle
\thispagestyle{fancy}
\noindent This program is a game, based on Snakes and Ladders and The Goose Game, but with more features.\\
The basic game is played on a board divided into tiles and has at least two players that move by a number of steps, decided by a dice roll. Tiles can have effects, which will be activated once a player lands on them. These effects can be positive or negative for the player. The player who reaches the end of the path first, wins the game.\\
\\
The main features will be:
\begin{itemize}
	\item [-] Board and game information displayed with tkinter/ turtle
	\item [-] Variable board length
	\item [-] Variable dice faces (range of the dice)
	\item [-] Randomly generated boards (random effect tile placing)
	\item [-] Difficulty settings (number and harshness of effects)
	\item [-] Save games
	\item [-] Statistics about the game
	\item [-] Customisation of pawns
	\item [-] Themes
	\item [-] New effects
	\item [-] Hide/ show tile effects\\
\end{itemize}
Possible tile effects will include:
\begin{itemize}
	\item [-] Moving back or forward by $x$
	\item [-] Skipping or gaining $x$ turns
	\item [-] Going to another player position
	\item [-] Going to a specific tile
	\item [-] Dice range shrank or expanded by $x$ for $y$ turns
	\item [-] Dice range shifted or expanded by $x$ for $y$ turns
	\item [-] Dice range stuck at $x$ for $y$ turns
\end{itemize}
\end{document}