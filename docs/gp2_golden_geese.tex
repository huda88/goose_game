\documentclass[11pt]{article}
\usepackage{fancyhdr}
\pagestyle{fancy}
\rhead{\today}
\lhead{}
\cfoot{}
\renewcommand{\headrulewidth}{0.4pt}
\renewcommand{\footrulewidth}{0.4pt}
\title{\textbf{The Goose Game, reimagined}\\High Level Design}
\author{Paulo Aurecchia, Samuele Decarli, Irene Jacob, Julia Traiba}
\date{}

\begin{document}

\maketitle
\thispagestyle{fancy}
\noindent The program will be divided in three branches: model, visual and control. Model handles the logic of the game, visual handles the graphics and user input, while control serves as an interface between the two and handles the game in general.\\

\section*{Model}
\subsection*{Data Structures}
\begin{itemize}
	\item [-] There is a dictionary representing the board, e.g. \{ 0: ('effect',), 1: ('effect1', value), ... \}
	\item [-] Another dictionary represents the players, e.g. \{1: [position, [dice], \{current\_effects\}, health] \}
\end{itemize}

\subsection*{Functions}
\begin{itemize}
	\item [-] \textit{board\_generator(length, dice, effects)} generates the board.
	\item [-] \textit{player\_generator(n\_players, dice)} generates the players.
	\item [-] All effects are functions that modify the player dictionary, based on where the player lands
\end{itemize}

\pagebreak

\section*{Control}
\subsection*{Functions}
\begin{itemize}
	\item [-] \textit{game\_handler()} runs the whole game.
	\item [-] \textit{turn(players, current\_p, board)} handles a turn.
	\item [-] Generally every function that passes data between GUI and model. 
\end{itemize}


\section*{Visual}
\subsection*{Data Structures}\
\begin{itemize}
	\item [-] inputs given by the user (string)
	\item [-] board (dictionary having number of tile as key and the actual tile as value)
	\item [-] players (dictionary with the number of the player as key and the coordinates as value)
	\item [-] graphical elements that have to be ``connected'' (for example, a box and its text that have to move together) (list)
\end{itemize}


\subsection*{Functions}
We will use Tkinter to implement the GUI of the game.
Below is a list of the basic functions we are planning to implement:
\begin{itemize}
	\item [-] function to generate 2 windows -- inputs and main window
	\item [-] function to return user inputs (number of players, board length, dice faces, difficulty level)
	\item [-] function to create all the buttons and show and modify stats
	\item [-] functions to create, and move players
	\item [-] function to generate and modify the board
	\item [-] function to colour tiles according to their effects, and animate certain effects
	\item [-] function to show winner
\end{itemize}

\end{document}
